% !TeX root = ../main.tex
% Add the above to each chapter to make compiling the PDF easier in some editors.

\chapter{Introduction}\label{chapter:introduction}

Time-series can be found in many places. Any monitoring task will produce time-series data.  A time-series is a sequence of data points in time order.  Relevant domains for time-series data include Internet of Things,  performance engineering of computer systems, and cloud computing. Across domains, there are various analytic primitives for time-series: forecasting, anomaly and outlier detection, clustering and classification.

A not as common task is the generation. Generating time-series can be useful in various scenarios. It can be used for testing purposes, for example, to simulate the input of a sensor in a system. The generating model can also serve as an input for a larger model. Moreover, the generated data can be used to simulate load to benchmark a system, which is an application that this paper will look into. 

The time-series generation approaches of choice in this paper are Hidden Markov Models and Singular Spectrum Analysis. Both are powerful approaches that are feasible to implement. They are also completely different providing a meaningful comparison. 

This paper presents the development of a time-series generation tool using these approaches and it will explore benchmarking as an example application of generated time-series. 