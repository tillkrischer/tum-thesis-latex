% !TeX root = ../main.tex
% Add the above to each chapter to make compiling the PDF easier in some editors.

\chapter{Evaluation of the Approaches}\label{chapter:evaluation}

Having looked at the two approaches of generating time-series in detail in both theory and implementation, we can evaluate their strengths and weaknesses and compare them to each other. 

\section{HMM}

We first look at Hidden Markov Models. HMMs are a robust method of analyzing time series. In Baum-Welch we have a solid algorithm for learning parameters that can deal with a lot of input data, including multiple separate time-series. It can even be implemented in a distributed fashion. 

By assuming an underlying model of the data HMMs are well suited for the generation task. We can use learned parameters or specify parameters manually and achieve useful results. Moreover, a single set of parameters can generate infinitely many time-series. 

The main downside of HMMs is also their defining characteristic, the Markov Property. Every output only depends on its predecessor. That constrains the approach and makes it very hard to maintain any state. Theoretically, to have n bits of state in a HMM we would need $2^n$ hidden states. The Markov Property also makes it impossible to model any kind of seasonality with HMMs. 

\section{SSA}

The weakness of HMMs, the handling of seasonality, is the strength of Singular Spectrum Analysis. Given a time-series, SSA will often automatically, that is without any knowledge of the series structure, identify trend and seasonality components. And based on those can continue the series using forecasting.

However, SSA is fundamentally ill-suited for a straight generation task, simply because it is not a generative model. We can only generate by forecasting an existing time-series. This means that for one input time-series there is only one generated time-series, with minor variations when adjusting the parameters of windows length and number of components.

Finally, SSA does not deal well with data where there is no seasonality or stationary components to extract. In those cases, the forecast do not make sense. In the worst case, it is completely off because of the recurrent nature of the forecast that makes errors compound as you go further along.