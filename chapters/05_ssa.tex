% !TeX root = ../main.tex
% Add the above to each chapter to make compiling the PDF easier in some editors.

\chapter{Singular Spectrum Analysis}\label{chapter:ssa}

The second approach used in the time series generation tool is Singular Spectrum Analysis (SSA). SSA is a method of time-series analysis with very different properties to Hidden Markov Models, which provides an interesting contrast. The applications of SSA include time-series decomposition, filtration, and forecasting \parencite{golyandina2013singular}.

SSA differs from many approaches of time-series analysis including HMMs in that there is no model assumed for the time series beforehand. Still, SSA can identify features of a time series like trends and periodicities. The structure of these components is defined as a linear recurrence relation, which is similar but not the same as an autoregressive model \parencite{golyandina2014basic}.

The main task of SSA is to decompose a time-series into a sum of components. It does this without any prior knowledge of the structure of the time series. 

\section{SSA Decomposition Algorithm}

Let $X$ be a time series of length $N$. The first step of the decomposition algorithm is to choose a window length $L$. Where $1 < L < N$. 

\begin{equation}
   X = (x_1, x_2, \ldots ,x_N)^T
\end{equation}

Next is the embedding step in which we create the so-called trajectory matrix. The trajectory matrix $T$ is a sequence of lagged vectors of size L. 

\begin{equation}
   T = 
\left(\begin{array}{ccccc}
x_{1} & x_{2} & x_{3} & \dots & x_{N-L+1} \\
x_{2} & x_{3} & x_{4} & \dots & x_{N-L+2} \\
x_{3} & x_{4} & x_{5} & \dots & x_{N-L+3} \\
\vdots & \vdots & \vdots & \ddots & \vdots \\
x_{L} & x_{L+1} & x_{L+2} & \dots & x_{N}
\end{array}\right)
   \label{eq:ssa-trajectory}
\end{equation}

Note that both the rows and columns of the trajectory matrix are subseries of X and that the values on the anti-diagonal are all the same. A matrix of this form is called a Hankel matrix. 

Next is the decomposition step. The idea is to find an orthonormal basis ${P_i}^L$ and decompose the trajectory matrix like this: 

\begin{equation}
   \begin{aligned}
   T&=\sum_{i=1}^{L} P_{i} Q_{i}^{\mathrm{T}}=\mathrm{X}_{1}+\ldots+\mathrm{X}_{L} \\
   \text{where} \;\; Q_i &= T^TP_i
   \end{aligned}
   \label{eq:ssa-decomposition}
\end{equation}

There are multiple choices for a basis. In this case, we choose ${P_i}^L$ to be the eigenvectors of $TT^T$. These can be obtained using Singular Value Decomposition. 

Given a matrix $A$ Singular Value decomposition returns matrices $U$, $V$, and $S$ such that $A = USV^T$. Matrices $U$ and $V$ are orthogonal matrices, meaning their product with themselves is the identity matrix. Matrix $S$ is a diagonal matrix. The columns of $U$ are the eigenvectors of $AA^T$ and the columns of $V$ are the eigenvectors of $A^TA$. The entries of $S$ are square roots of the eigenvalues of $AA^T$ or $A^TA$. The order of the eigenvalues in S is from highest to lowest. 

The Singular Value decomposition of the trajectory matrix $T$ can be expressed like this: 

\begin{equation}
T=\sum_{i} \sqrt{\lambda_{i}} U_{i} V_{i}^{\mathrm{T}}
\end{equation}

With $\lambda_i$ being the i-th eigenvalue.

This is is the format we were looking for in Equation \eqref{eq:ssa-decomposition}. We have already defined $P_i=U_i$ which makes $Q_i = \sqrt{\lambda_i}V_i$. This completes the decomposition step \parencite{golyandina2014basic}.

\section{Eigentriple Grouping}

The results of the Singular Value Decomposition $(\lambda_i, U_i, V_i)$ are referred to as eigentriples. One eigentriple does not necessarily correspond to one component of the time series. Components that SSA can identify include sine waves, trends, and seasonality. These all correspond to a group of eigentriples. 

Formally we pick all the eigenvectors with eigenvalues greater than 0:

\begin{equation}
  d=\max \left\{j: \lambda_{j} \neq 0\right\} 
\end{equation}

Then consider the set of indices $\{1, \ldots, d\}$ and group it into $m$ subsets $I_1, \ldots, I_m$. Then use these to split up the trajectory matrix $T$. 

\begin{equation}
   \begin{aligned}
      T&=\mathbf{X}_{I_{1}}+\ldots+\mathbf{X}_{I_{m}} \\
      \text{where} \;\; \mathbf{X}_{I}&=\sum_{i \in I} \mathbf{X}_{i}
   \end{aligned}
\end{equation}
  
If the goal of the SSA analysis is to extract components like sine waves from the time series, a lot of care has to be taken when choosing these groupings. One can look at the so-called w-correlation matrix to figure out which components are correlated and which are separable. Or you can use known facts like sinusoid components being always comprised of two sine wave components phase-shifted by $\pi/2$ \parencite{golyandina2014basic}.

For the time series generation application, these considerations do not apply. It is enough to just obtain an approximation of the time series, which can be obtained in a straight forward way. We know that the eigentriples obtained from the Singular Value Decomposition come in order of eigenvalue. That means that the components, which contribute most to the time-series, come first. By just picking an $n$ and grouping together the first $n$ eigentriples, we receive a good approximation of the original time-series.

\section{Diagonal Averaging}

So far we have only operated on the trajectory matrix and broken it up into components. This means that all the components are in matrix form. To return them to the time-series format we apply diagonal averaging. 

As the name suggests it works by averaging over the anti-diagonals of a matrix. Consider a $L\times K$ matrix. The anti-diagonals are defined like this: 

\begin{equation}
A_{s}=\{(l, k): l+k=s+1,1 \leq l \leq L, 1 \leq k \leq K\}
\end{equation}

To convert matrix $X$ into time-series $\tilde{x}$ we take the averages across these anti-diagonals. The time series $\tilde{x}$ is called the reconstructed time-series. 

\begin{equation}
    \widetilde{x}_{s}=\sum_{(l, k) \in A_{s}} X_{l k} /\left|A_{s}\right|
\end{equation}


\begin{figure}
   \begin{equation*}
      X = 
      \begin{pmatrix}
         1 & 2 & 3 & 4 \\
         5 & 6 & 7 & 8 \\
         9 & 10 & 11 & 12
      \end{pmatrix}
   \end{equation*}
   \begin{equation*}
      \begin{aligned}
      \widetilde{x} &= (1, \frac{2+5}{2}, \frac{9 + 6 + 3}{3}, \frac{10 + 7 + 4}{3}, \frac{11 + 8}{2}, 12) \\
      &= (1, 3.5, 6, 7, 9.5, 12)
      \end{aligned}
   \end{equation*}
\caption{Diagonal Averaging Example}
\label{fig:diagonal-averaging}
\end{figure}

Figure \ref{fig:diagonal-averaging} shows an example of diagonal averaging. 

\section{Forecasting}

With the described algorithm we can use SSA to identify the components of a time series and obtain an approximation. However, this does not help with the goal of time-series generation. SSA, as mentioned before, does not assume an underlying model of the time series. Thus, we can not go the route of Hidden Markov Models where we generate a series from model parameters. 

What we can do however is the forecasting of a given time series. While this does not give a brand new time-series, it allows us to generate a infinite time-series based on a given finite time-series.

There are multiple approaches to forecast using SSA. The most straight forward one is Recurrent Forecasting. It involves setting up a recurrence relation for the time series. Recurrence relation means a value is computed using its predecessors. With this, we can continue the given time series by computing new values. 

We take the same set of eigentriples $I$ that we picked during the eigentriple grouping. Let $P_i, \; i \in I$ be the eigenvectors of the eigentriple group. We split them up in the following way: $\underline{P_i}$ is everything but the last coordinate of $P_i$ and $\pi_i$ is the last coordinate of $P_i$. 

Define the vector $R = (a_{L-1}, \dots, a_1)^T$ like this:

\begin{equation}
   R=\frac{1}{1-\sum_{i \in I} \pi_{i}^{2}} \sum_{i \in I} \pi_{i} \underline{P}_{i}
   \label{eq:r-def}
\end{equation}

Finally, the forecasted time series $y$ can be computed. The recursive definition looks like this:

\begin{equation}
  y_{i}=\left\{\begin{array}{ll}
\widetilde{x}_{i} & \text { for } i=1, \ldots, N \\
\sum_{j=1}^{L-1} a_{j} y_{i-j} & \text { for } i=N+1, \ldots, N+M
\end{array}\right. 
\end{equation}

\newpage

Here $\tilde{x}$ refers to the reconstructed time-series obtained in the diagonal averaging step, which makes up the first part of the series. The second part is the forecast based on the recurrence relation \parencite{golyandina2014basic}.

\section{Parameter Concerns}

There are two parameters when generating time series with SSA using this approach. The first is the window length $l$. It should be sufficiently large. The larger the window the more information there is to identify components. When trying to find seasonal components the window length should be divisible by the suspected seasonality. On the other hand, the window length should not be larger than $N/2$. $N$ being the length of the original time-series. The size of the trajectory matrix is $L \times (N-L+1)$. If $L$ is to large the trajectory matrix has too few columns. 

The other parameter $K$ is the number of eigentriples used in the reconstruction. Picking a smaller $K$ will result in a smoothed reconstruction, while a bigger $K$ will result in a reconstruction very close to the original time-series. By design, the latter components only have a small contribution to the result, so overfitting by picking a too large $K$ is not a concern. The upper bound for $K$ is the window length $L$. This is because the trajectory matrix has $L$ rows and therefore, can not have more than $K$ eigenvectors. 